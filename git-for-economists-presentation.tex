\documentclass{beamer}

\mode<presentation>
{
  \usetheme{default}      % or try Darmstadt, Madrid, Warsaw, ...
  \usecolortheme{default} % or try albatross, beaver, crane, ...
  \usefonttheme{default}  % or try serif, structurebold, ...
  \setbeamertemplate{navigation symbols}{}
  \setbeamertemplate{footline}[frame number]
  \setbeamertemplate{caption}[numbered]
} 

\usepackage[english]{babel}
\usepackage[utf8x]{inputenc}

\usepackage{graphicx}
\usepackage{subcaption}
\usepackage{hyperref}
\usepackage{verbatim}
\usepackage{textcomp}

\title[Git for Economists]{Git: A Guide for Economists}
\author{Frank Pinter}
\date{22 February 2019}

\AtBeginSection[]
{
  \begin{frame}
    \frametitle{Table of Contents}
    \tableofcontents[currentsection]
  \end{frame}
}

\begin{document}

\begin{frame}
  \titlepage
\end{frame}

\begin{frame}{Outline}
  \tableofcontents
\end{frame}

\section*{Introduction}

\begin{frame}{What is version control?}
Version control is a way to keep track of changes to code, text, and documents. And data and outputs.
\begin{itemize}
\item It gives you an organized revision history
\item It lets you experiment without fear
\item It lets you go back and forth between many different versions of the same file, and see a list of the differences
\item It makes (the technical aspects of) collaboration a breeze
\item It lets you and your collaborators work on different versions and then merge them
\end{itemize}
\end{frame}

\begin{frame}{What is Git?}
\begin{itemize}
\item Git is a program that does version control
\item It is the most popular version control program in software development
\item It is easy to set up and get started
\item There are many programs that add intuitive interfaces on top of Git
\item Git integrates seamlessly with online collaboration tools like GitHub and GitLab
\end{itemize}
\end{frame}

\section{The importance of version control}

\begin{frame}{Life without version control}
Do you keep every variant of every program you ever wrote on a project?
\begin{itemize}
\item The code and the outputs?
\item What if you discover a coding error? Which versions are right?
\item How do you organize all the files?
\end{itemize}
Or, worse, do you only keep the latest thing you tried?
\begin{itemize}
\item What if you introduce a new mistake?
\item What if you're experimenting and you accidentally keep the experiment in?
\end{itemize}
\end{frame}

\begin{frame}{Version control 0.1: putting dates on things}
Does this look familiar?
\texttt{run\_regs\_11\_17\_2018\_v4\_final\_final.do}
\\
``Not one piece of commercial software you have on your PC, your phone, your tablet,
your car, or any other modern computing device was written with the `date and initial' method.'' (Gentzkow and Shapiro)
\end{frame}

\begin{frame}{Version control 0.2: Dropbox}
\begin{itemize}
\item Dropbox keeps a crude version history.
\begin{itemize}
\item But there are no labels or comments, and it's not easy to see the differences between files.
\item So if you want to dig up ``the version where I had that other variable'' you have to manually look through a bunch of versions.
\item And good luck if you changed two scripts, not just one.
\end{itemize}
\item Dropbox lets you and your collaborators stay in sync.
\begin{itemize}
\item What if you and your coauthor try to change the same script at the same time?
\item What if you are trying one change and, at the same time, your coauthor is trying a different change?
\end{itemize}
\end{itemize}
\end{frame}

\begin{frame}{Version control 0.2: Dropbox}
A Post It note spotted on a grad student's desk:
\begin{quote}
	Don't forget! At 10:18 am on November 17th, we changed the specification to add new variable.
\end{quote}
Don't live this way.
\end{frame}

\section{Getting started on a project}

\begin{frame}{Why use Git?}
Git is the dominant version control system today. There are others, but they're generally more work with no benefit.
\end{frame}

\begin{frame}{Getting started with Git}
\begin{enumerate}
\item \href{https://git-scm.com/book/en/v2/Getting-Started-Installing-Git}{Install Git} (Linux, Mac, Windows)
\item Git comes with a command line interface (powerful!). You might want to add a graphical interface to make things easier:
\begin{itemize}
\item \href{https://www.gitkraken.com/}{GitKraken}
\begin{itemize}
\item The examples in this presentation use GitKraken
\end{itemize}
\item \href{https://desktop.github.com/}{GitHub Desktop}
\item \href{https://support.rstudio.com/hc/en-us/articles/200532077-Version-Control-with-Git-and-SVN}{RStudio (for R projects)}
\end{itemize}
\end{enumerate}
\end{frame}

\begin{frame}{Getting started with your project}
If you're starting your own project in GitKraken:
\begin{itemize}
\item Open GitKraken and select  ``Start a local project'' or ``Start a hosted project'' (will get into this later)
\item Choose the name and the local directory to use
\item Start working in the directory
\end{itemize}
If your coauthor started a project, added you to it online, and you're putting it on your own computer:
\begin{itemize}
\item Open GitKraken and go to File \textrightarrow Clone
\item Select the service (GitHub/GitLab/Bitbucket), log in if necessary, and select the project from the list
\item Choose the local directory to save it to
\end{itemize}
\end{frame}

\begin{frame}{What actually is the Git repository?}
\begin{itemize}
\item The Git local repository is associated with a particular directory
\item Open the directory in your Git interface to see your options
\item Git stores all its workings in that directory in a hidden subfolder called ``.git''
\item Corollary: \href{https://stackoverflow.com/questions/19305033/why-is-putting-git-repositories-inside-of-a-dropbox-folder-not-recommended}{don't use Git inside a Dropbox shared folder!}
\item Sharing is what remote repositories and hosting services are for (more on this later)
\end{itemize}
\end{frame}

\begin{frame}{What should I include?}
\begin{enumerate}
\item At a minimum:
\begin{itemize}
\item Code (\texttt{.do}, \texttt{.R}, \texttt{.m}, \texttt{.jl}, and so on)
\item Text files (\texttt{.txt})
\item \LaTeX \, documents (\texttt{.tex})
\end{itemize}
\item I also recommend:
\begin{itemize}
\item Raw \texttt{.csv} datasets, if small (\textless 10 MB)
\end{itemize}
\item These are binary files, so you can't see differences between versions. I recommend including them anyway.
\begin{itemize}
\item PDF files
\item Word, Excel, PowerPoint files
\end{itemize}
\item Some people also include all datasets.
\begin{itemize}
\item Note that GitHub doesn't allow files larger than 100 MB, or projects with total size larger than 1 GB.
\end{itemize}
\end{enumerate}

For datasets, look into \href{https://git-lfs.github.com/}{Git Large File Storage}.
\end{frame}

\begin{frame}[fragile]{What should I exclude?}
In order to avoid driving your coauthors crazy, you \textbf{must} tell Git to ignore the junk files using a file called \textit{.gitignore}. It looks like this:

\begin{verbatim}
# Junk created by LaTeX
*.synctex.gz
*.out
*.log
# Junk created by R
.RData
# Junk created by Python
*.pyc
\end{verbatim}

Best practice: use \textit{.gitignore} to explicitly exclude \textit{everything} that you don't want to include, and commit \textit{.gitignore} like any other regular file.

GitHub maintains a \href{https://help.github.com/articles/ignoring-files/}{list of standard \textit{.gitignore}} files for many common languages.
\end{frame}

\section{Using Git}

\begin{frame}{The Git model}

\begin{enumerate}
\item You do work in your \textbf{working directory}.
\item Then you add it to your \textbf{staging area}.
\item Once you've staged all your changes for one discrete task, \textbf{commit} a snapshot of the staging area.
\item If you have a remote repository, \textbf{push} your commit to the remote repository.
\end{enumerate}

\includegraphics[width=\textwidth]{git-model.pdf}
\end{frame}

\begin{frame}{Commits: saving a snapshot}
What is ``one discrete task''? A collection of changes, across multiple files, that does \textit{one thing}. Examples:
\begin{itemize}
\item Change the formatting of a variable from string to numeric, and treat it properly across multiple scripts
\item Change your regression specification in code, in the output, and in your paper and supporting documentation
\item Add a new function, and tests for that function
\end{itemize}
These form one \textbf{commit}, which you annotate with a detailed \textbf{commit message}. Examples:
\begin{itemize}
\item ``Change the formatting of start date variable from string to Stata date format''
\item ``Add year dummies to regression specification''
\end{itemize}
The more detail, the more your future self will thank you.
\end{frame}

\begin{frame}{Commits}
\includegraphics[width=200px]{screenshots/committing.png}
\end{frame}

\begin{frame}{Run tests before you commit}
Your code should run properly when you commit.
\begin{itemize}
\item No runtime errors
\begin{itemize}
\item Test this by running all code that changed, and everything that depends on it
\item Makefiles automate this process
\item Only skip if you are sure you didn't change anything important
\end{itemize}
\item No compilation errors (including \LaTeX)
\item All your tests should pass
\item Output should be consistent with what you've written
\begin{itemize}
\item Don't report a negative regression coefficient, and write in words that the estimated coefficient is positive
\end{itemize}
\end{itemize}
But it's better to have frequent commits (that might have small mistakes) than to have giant, infrequent commits.
\end{frame}

\begin{frame}{Viewing changes when committing}
Git easily lets you see what changed in \LaTeX, code (not images/PDFs/most datasets). Review this when staging!
\includegraphics[width=\textwidth]{screenshots/diff.png}
{\small (This is the GitKraken interface, but it looks similar in any other interface)}
\end{frame}

\begin{frame}{Viewing history}
GitKraken shows you a list of past changes. You can also see just the history of a particular file:
\includegraphics[width=\textwidth]{screenshots/history-diff.png}
\end{frame}

\begin{frame}{Undoing history}
What happens when a commit was a mistake? Revert it, to make a new commit that undoes it.

In GitKraken, right click on the commit line and select ``Revert commit''.
\includegraphics[width=\textwidth]{screenshots/revert-history.png}
\end{frame}

\begin{frame}{Branches: trying things out}
\href{https://git-scm.com/book/en/v2/Git-Branching-Branches-in-a-Nutshell}{Branches} are the most powerful part of Git
\begin{itemize}
\item By default, all the work you do goes into the ``master'' branch
\item Want to experiment? Start a new branch
\item You can switch between branches, and make commits to either branch
\begin{itemize}
\item There is a catch: if you don't include your intermediate/final datasets in Git, you may need to re-make them when you switch
\end{itemize}
\item If your experiment works out, commit and merge back into the master branch
\begin{itemize}
\item If there are conflicts between the commits you've made on the two branches, Git will ask you to resolve them
\item This is easiest with a graphical interface like \href{https://support.gitkraken.com/working-with-repositories/branching-and-merging/}{GitKraken}
\item This can't be done with binary files (PDF, images, Word, Excel). You'll just have to decide which one to keep.
\end{itemize}
\item If your experiment doesn't work out, delete the new branch painlessly
\end{itemize}
\end{frame}

\begin{frame}{Keeping it local vs. using a remote repository}
Git doesn't require a remote repository. You can run it 100\% on your computer, with no connection to an outside server.
\begin{itemize}
\item This is useful if you have restrictions on your code (for example, you work with confidential health data)
\begin{itemize}
\item Ask me if you have questions about using Git this way on the NBER cluster
\end{itemize}
\item But a remote repository helps you keep things backed up seamlessly, and lets you collaborate with others
\item You can push all your branches to the remote repository, or only some of them
\end{itemize}

\end{frame}

\section{Using Git for collaboration}

\begin{frame}{Remote repository}
The remote repository is on a server, and holds a record of your commits and branches
\begin{itemize}
\item You \textbf{push} to the remote repository to save all your commits
\item You \textbf{pull} from the remote repository to load all new commits
\item Always commit before pushing or pulling
\item If what you're doing is an experiment, make a new branch to avoid any trouble for your coauthor
\item As with branches, if there are conflicts between your commits and your coauthor's commits, Git will ask you to resolve them
\end{itemize}
\end{frame}

\begin{frame}{Hosting services}
To collaborate, you'll need a service to host your remote repository.
\begin{itemize}
\item Here are a few:
\begin{itemize}
\item \href{https://github.com/}{GitHub} (most popular)
\item \href{https://gitlab.com/}{GitLab}
\item \href{https://bitbucket.org/}{Bitbucket}
\end{itemize}
\item You can choose \textit{public} (published for the world to see) or \textit{private} (best for research in progress)
\item Most services have restrictions on private repositories
\item It's easy to use one service for one project, and another service for another project
\item All these services have nice web interfaces for managing your project
\item Some also have integrated task management systems
\end{itemize}
\end{frame}

\section*{Conclusion}

\begin{frame}{Learning the command line}
\begin{itemize}
\item There are many more features that are best accessed from the Git command line
\item And in some situations (like the NBER servers) you don't have a choice
\item A fantastic resource for learning the command line: Jes\'us Fern\'andez-Villaverde's notes on Git (\url{https://www.sas.upenn.edu/~jesusfv/Chapter_HPC_5_Git.pdf})
\item See also his class!
\end{itemize}
\end{frame}

\begin{frame}{Conclusion}
\begin{itemize}
\item At its simplest, Git is a way to keep track of the history of your work, and easily go back to past versions
\item But it can be so much more!
\item Experiment without fear
\item Collaborate with far less back-and-forth
\item The best way to learn Git: use it!
\end{itemize}
\end{frame}

\begin{frame}{Further reading}
\begin{itemize}
\item Again, Jes\'us Fern\'andez-Villaverde's notes on Git (\url{https://www.sas.upenn.edu/~jesusfv/Chapter_HPC_5_Git.pdf})
\item Hadley Wickham's book on writing R packages. The chapter on Git and GitHub (\url{http://r-pkgs.had.co.nz/git.html}) is well-written and not specific to R.
\item If you want to drill down on workflow, see the tutorial ``Understanding the GitHub flow'' (\url{https://guides.github.com/introduction/flow/})
\end{itemize}
\end{frame}

\end{document}
